Throughout this solution manual, we will use the same notation (and convention) as in the book, with probably a little to none changes.

For your convenience, it is recommended to search your question via whatever your browser provides (e.g. F3). The format of questions are \textit{Chapter}(in roman).\textit{Section.Question}.

In the following, categories are denoted using the \textsf{Sans-serif} font, e.g. $\mathsf{Set}$.

\section{}
\begin{problem}{I.1.1}
Locate a discussion of Russel's paradox, and understand it.
\end{problem}

\begin{problem}{I.1.2}
Prove that if $\sim$ is an equivalence relation on a set $S$, then the corresponding family $\mathscr{P}_\sim$ defined in $\S 1.5$ is indeed a partition of $S$.
\end{problem}
\begin{pf}
The union of such class must contain $S$ by definition, as at worse the elements can be in the equivalence class formed by themselves. It suffices to check disjointness: If $a \in [x], a \in [y]$ but $x \nsim y$, then transitivity implies $x \sim a, a \sim y \Rightarrow x \sim y$, a contradiction. 
\end{pf}

\section{}
\begin{problem}{I.2.1}
How many different bijection are there between a set $S$ with $n$ elements and itself?
\end{problem}
\begin{sol}
The first number has $n$ choices; to make the map a bijection, the next number has only $(n-1)$ choices remaining. By continuing choosing, we have $n!$ different bijections.
\end{sol}

\begin{problem}{I.2.5}
Formulate a notion of \emph{epimorphism}, in the style of the notion of \emph{monomorphism}, and prove a result analogous to Proposition 2.3, for epimorphisms and surjections.
\end{problem}
\begin{sol}
Epimorphism are \emph{right-cancelable}; that is,
\begin{center}
A function $f : A \to B$ is a epimorphism if for all sets $Z$ and all functions $\beta, \beta' : Z \to A$, 
\[
\beta \circ f = \beta' \circ f \Longrightarrow \beta = \beta'.
\]
\end{center}
We shall prove the following:

\noindent \textbf{Proposition. }\textit{A function is surjective if and only if it is an epimorphism.} \\
\textit{Proof. } \\
$(\Rightarrow)$ Let $f$ be surjective. By Proposition I.2.1, a surjective function has a right-inverse, which we call it $g$. Then if $\beta, \beta' : B \to Z$ are arbitrary function such that $\beta \circ f = \beta' \circ f$, then by composition with $g$ we obtain
\[
(\beta \circ f) \circ g=  (\beta' \circ f) \circ g \Rightarrow \beta \circ (f \circ g)=  \beta' \circ  (f \circ g) \Rightarrow \beta \circ id_A = \beta' \circ id_A \Rightarrow \beta = \beta'
\]
as desired. \\
$(\Leftarrow)$ Let $f$ be an epimorphism. We need to consider some special $\beta:B \to Z$ so we can prove the assertion. We done this by "labeling": define
\[
\beta(b) = \begin{cases}
    1, &b \in \im f \\
    0, &b \notin \im f
\end{cases}, \quad \beta'(b) = 1
\]
Then since 
\[
\beta \circ f = \beta' \circ f  \Rightarrow \beta = \beta'
\] 
this implies that beta receives \emph{only} values in $\im f$, so $\im f \supseteq B$. Since we have $\im f \subseteq B$ clearly for any function $f$, we conclude that $\im f = B$, which is the definition of surjectivity.
\end{sol}

\section{}
\begin{problem}{I.3.1}
Let \textsf{C} be a category. Consider a structure $\mathsf{C}^{op}$ with
\begin{itemize}
\setlength\itemsep{0pc}
\item $\text{Obj}(\mathsf{C}^{op}) = \text{Obj}(\mathsf{C})$;
\item for $A, B$ objects of $\mathsf{C}^{op}$, $\Hom_{\mathsf{C}^{op}}(A,B) := \Hom_\mathsf{C}(B,A)$.
\end{itemize}
Show how to make this into a category.
\end{problem}
\begin{sol}
For $f \in \Hom_{\mathsf{C}^{op}}(A,B), g \in \Hom_{\mathsf{C}^{op}}(B,C)$, define the composite of morphisms by
\[
g \circ f := fg     
\]
where $fg$ is defined in the sense of the category $\mathsf{C}$. Now we check the definition of category:
\begin{itemize}
\setlength\itemsep{0pt}
\item $1_A$ exists as $\Hom_{\mathsf{C}^{op}}(A,A) := \Hom_\mathsf{C}(A,A) \ni 1_A$; 
\item The composition works as intended: the map on the right is a morphism from $C$ to $A$;
\item The composite law is checked as
\[
(h \circ g) \circ f = gh \circ f = f(gh) = (fg)h = h \circ fg = h \circ (g \circ f);
\]
\item Identity morphism work as intended:
\[
1_A \circ f = f1_A = f, \quad f \circ 1_A = 1_Af = f.	
\]
\end{itemize}
\end{sol}

\begin{problem}{I.3.11}
Draw the relevant diagrams and define composition and identities for the category $\mathsf{C}^{A,B}$ mentioned in Example 3.9. Do the same for the category $\mathsf{C}^{\alpha, \beta}$ mentioned in Example 3.10. 
\end{problem}
\begin{sol}
By reversing the arrow of $\mathsf{C}_{A,B}$, we obtain:
\begin{itemize}
\setlength\itemsep{0pt}
\item Objects of this category are diagrams
\[
\begin{tikzcd}[column sep=small, row sep=small]
A \arrow[dr, "f"] \\
& Z \\
B \arrow[ur, "g"']
\end{tikzcd}
\]
\item morphisms are 
\[
\begin{tikzcd}[column sep=small, row sep=small]
A \arrow[dr, "f_1"] \\
& Z_1 \\
B \arrow[ur, "g_1"']
\end{tikzcd}
\longrightarrow
\begin{tikzcd}[column sep=small, row sep=small]
A \arrow[dr, "f_2"] \\
& Z_2 \\
B \arrow[ur, "g_2"']
\end{tikzcd}    
\]
which are commutative diagrams
\[
\begin{tikzcd}[column sep=small, row sep=small]
A \arrow[dr, "f_2"] \arrow[drr, bend left, "f_1"] \\
&Z_1 \arrow[r, "\sigma"] &Z_2 \\
B \arrow[ur, "g_2"'] \arrow[urr, bend right, "g_1"']
\end{tikzcd}.
\]
\end{itemize}
For the case $\mathsf{C}^{\alpha, \beta}$:
\begin{itemize}
\item Objects are diagrams
\[
\begin{tikzcd}[column sep=small, row sep=small]
&A \arrow[dr, "f"] \\
C\arrow[ur, "\alpha"]\arrow[dr, "\beta"'] & &Z \\
&B \arrow[ur, "g"']
\end{tikzcd}  
\]
\item morphisms are 
\[
\begin{tikzcd}[column sep=small, row sep=small]
&A \arrow[dr, "f_1"] \\
C\arrow[ur, "\alpha"]\arrow[dr, "\beta"'] & &Z_1 \\
&B \arrow[ur, "g_1"']
\end{tikzcd}     
\longrightarrow
\begin{tikzcd}[column sep=small, row sep=small]
&A \arrow[dr, "f_2"] \\
C\arrow[ur, "\alpha"]\arrow[dr, "\beta"'] & &Z_2 \\
&B \arrow[ur, "g_2"']
\end{tikzcd}  
\]
which are commutative diagrams
\[
\begin{tikzcd}[column sep=small, row sep=small]
& A \arrow[rd, "f_1"] \arrow[rrd, bend left, "f_2"]  &  &   \\
C \arrow[ru, "\alpha"] \arrow[rd, "\beta"'] &  & Z_1 \arrow[r, "\sigma"] & Z_2 \\
& B \arrow[ru, "g_1"'] \arrow[rru, bend right, "g_2"'] &  &    
\end{tikzcd}.
\]
\end{itemize}
composition and identity are defined analogously as in Example 3.5. 
\end{sol}


\section{}
\begin{problem}{I.4.3}
Let $A,B$ be objects of a category \textsf{C}, and let $f \in \Hom_\mathsf{C}(A,B)$ be a morphism.
\begin{itemize}
\setlength\itemsep{0pt}
\item Prove that if $f$ has a right-inverse, then $f$ is an epimorphism.
\item Show that the converse does not hold, by giving an explicit example of a category and an epimorphism without a right-inverse.
\end{itemize}
\end{problem}
\begin{pf}
Let $g$ be the right inverse of $f$, i.e. $fg = 1$. Then for any morphism $h, h' \in \Hom_\mathsf{C}(B,Z)$,
\[
h \circ f = h' \circ f \Rightarrow h \circ f \circ g = h' \circ f \circ g \Rightarrow h \circ 1 = h' \circ 1 \Rightarrow h = h'
\]
showing that $f$ is an epimorphism. For a counterexample in which the converse does not hold, consider $\mathsf{C} = \Z$, objects are integers, and morphisms are the relation $\leq$ (c.f. p.p.27). Then
\[
f : 1 \to 2	
\]
is an epimorphism, but there are no right inverse for $f$, since there are no morphisms in $\Hom_\mathsf{C}(2,1)$.
\end{pf}

\section{}

\begin{problem}{I.5.1}
Prove that a final object in a category \textsf{C} is initial in the opposite category $\mathsf{C}^{op}$ (I.3.1).	
\end{problem}
\begin{pf}
Let $F$ be a final object in \textsf{C}, which means that the set $\Hom_\mathsf{C}(A, F)$ is a singleton for all $A \in \text{Obj}(\mathsf{C})$. Since
\[
\Hom_\mathsf{C}(A, F) = \Hom_\mathsf{C^{op}}(F, A)
\]
we have that $F$ is initial in $\mathsf{C}^{op}$.
\end{pf}


\begin{problem}{I.5.12}
Define the notions of \emph{fibered products} and \emph{fibered coproducts}, as terminal objects of the categories $\mathsf{C}_{\alpha, \beta}, \mathsf{C}^{\alpha, \beta}$ considered in Example 3.10 (cf. also I.3.11), by stating carefully the corresponding universal properties. 

As it happens, \textsf{Set} has both fibered products and fibered coproducts. Define these objects 'concretely', in terms of naive set theory.
\end{problem}
\begin{sol}
Fibered product is \emph{final} in $\mathsf{C}_{\alpha, \beta}$; that is, there are only one morphism in 
\[
\Hom\left(
\begin{tikzcd}[column sep=small, row sep=small]
& A \arrow[rd, "\alpha"] &   \\
Z \arrow[ru, "f_a"] \arrow[rd, "f_b"'] &  & C \\
& B \arrow[ru, "\beta"'] &  
\end{tikzcd},
\begin{tikzcd}[column sep=small, row sep=small]
& A \arrow[rd, "\alpha"] &   \\
F \arrow[ru, "i_a"] \arrow[rd, "i_b"'] &  & C \\
& B \arrow[ru, "\beta"'] &  
\end{tikzcd}
\right)    
\]
for any choice of the triple $(Z,f_a,f_b)$. Expand this to a diagram leads to the following universal property:

\textit{
The triple $(F, i_a:F \to A, i_b:F \to B)$ is universal in the sense that for every triple $(Z,f_a:Z \to A,f_b:Z \to B)$, there exists a unique morphism $\varphi: Z \to F$ such that the diagram}
\[
\begin{tikzcd}
Z \arrow[rd, "\exists!\varphi", dashed] \arrow[rdd, "f_b"', bend right] \arrow[rrd, "f_a", bend left] & & \\
& F \arrow[d, "i_b"] \arrow[r, "i_a"'] & A \arrow[d, "\alpha"] \\
& B \arrow[r, "\beta"']  & C                    
\end{tikzcd}    
\]
\textit{commutes.} Fibered product are also called \emph{pullback}.

Fibered coproduct is \emph{initial} in $\mathsf{C}^{\alpha, \beta}$. Following the same argument as above, we have the following universal property:

\textit{
The triple $(I, i_A:A \to I, i_B:B \to I)$ is universal in the sense that for every triple $(Z,f_A:A \to Z,f_B:B \to Z)$, there exists a unique morphism $\varphi: I \to Z$ such that the diagram}
\[
\begin{tikzcd}
C \arrow[r, "\alpha"] \arrow[d, "\beta"']         & A \arrow[d, "i_A"'] \arrow[rdd, "f_A", bend left] &   \\
B \arrow[r, "i_B"] \arrow[rrd, "f_B", bend right] & I \arrow[rd, "\exists!\varphi", dashed]   &   \\
& & Z
\end{tikzcd}    
\]
\textit{commutes}. Fibered coproduct are also called \emph{pushout}.

\textsf{Set} has fibered products: Let us define
\[
A \times_C B := I = \{(a,b):a\in A, b \in B, \alpha(a) = \beta(b)\}    
\]
with projections $i_a, i_b$. We check that this satisfy the universal property: define
\[
\varphi(z) := (f_a(z),f_b(z))    
\]
we check:
\begin{itemize}
\item $i_b \varphi = f_b$ (resp. $i_a \varphi = f_a$):
\[
i_b \varphi (z) = i_b (f_a(z),f_b(z)) = f_b(z)    
\]
\item $\alpha i_a = \beta i_b$:
\[
\alpha i_a (a,b) = \alpha(a) \overset{!}{=} \beta(b) = \beta i_b(a,b).
\]
note that ! is true since $I$ guarantees the existence of $b$.
\end{itemize}

\textsf{Set} also has fibered coproducts, but it's more complicated. We first define an equivalence relation: define
\[
R = \{(\alpha(x), 0) \sim (\beta(x), 1) : x \in C\}
\]
This gives an equivalence relation on $A \amalg B$, which gives a new structure $I = (A \amalg B)/\sim$. Let $i_A(a) = (a,0), i_B(b) = (b,1)$, then it is direct that $i_B \beta = i_A \alpha$. Now we define
\[
\varphi[i = (x,c)] = \begin{cases}
f_A(x) &\text{ if } c = 0 \\
f_B(x) &\text{ if } c = 1
\end{cases}    
\]
We need to check that it is well-defined, then it is direct that $\varphi \beta = f_B$ (resp. $\varphi \alpha = f_A$), proving the universal property. There are two cases to consider:
\begin{itemize}
\item Case $[(a,0)] = [(a',0)]$ (resp. $[(b,1)] = [(b',1)]$): If there are relations
\[
a = \alpha(x) \sim \beta(x) = \beta(x') \sim \alpha(x') = a'   
\]
then they evaluated to the same value since
\end{itemize}
\[
\varphi[(a,0)] = \varphi i_A(a) = \varphi i_A(\alpha(x)) = \varphi i_B(\beta(x)) = \varphi i_B(\beta(x')) = \varphi i_A(\alpha(x')) = \varphi i_A(a') = \varphi[(a',0)]
\]
\begin{itemize}	
\item Case $[(a,0)] = [(b,1)]$: If there are relations
\[
a = \alpha(x) \sim \beta(x) = b    
\]
then 
\[
\varphi[(a,0)] = \varphi i_A(a) = \varphi i_A(\alpha(x)) = \varphi i_B(\beta(x)) = \varphi i_B(b) = \varphi[(b,1)]	
\]
as desired.
\end{itemize}
By the above analysis, as all elements in the same equivalence class connects to the other by some chain
\[
a = \alpha(x_1) \sim \beta(x_1) = \beta(x_2) \sim \alpha(x_2) = \alpha(x_3) \cdots = b,
\]
and since every $\sim$ preserves the result, $\varphi$ is well-defined.
\end{sol}
